% \vspace{2in}

\begin{abstract}
This thesis addresses the problem of finding well-conditioned solutions to the Linear Factor Model for time series, satisfying $AA^\top=I$. It introduces the problem of Functional Multi-Objective Optimization Problem (fMOOP), in which functions are sought as solutions of Multi-Objective Optimization Problem. The thesis provides motivation for studying such problems and why they are important. It surveys different methodologies for solving MOOP in general and summarizes a few results on the condition number of matrices and functions. The thesis addresses important questions regarding such problems and analyzes different methods for finding well-conditioned solutions to fMOOP. Based on the analysis, the thesis selects a few methods and compares their experimental results to solve the problem of finding well-conditioned solutions to the Linear Factor Model for time series. Using the property of the space of matrices satisfying $AA^\top=I$, a set of extensions of methods are proposed and compared. The methods are implemented and the results show that the method which utilizes the property performs better compared to other methods. Hence, the thesis demonstrates that it is possible to design good extensions of known methods to solve the problem. This work highlights the importance of studying fMOOP and provides valuable insights into solving these problems.
 
% \newline\newline Will be completed when other contents are added

% \begin{itemize}
%     \item what is my research problem and objectives/goal [Done]
%     \item what methods did/can I used [Done]
%     \item what were the key results and arguments
%     \item conclusion
% \end{itemize}



% \paragraph{} Growing Hate-content on online platforms is a serious problem and has given rise to numerous  discussions about the need for regulation and monitoring of posted content. However using stringent measures such as account suspension or removal of the post somehow is a threat to the doctrine of free speech.   

% \paragraph{} Researchers worldwide have found that counter -narratives can effectively neutralize hate speech and incite constructive dialogue. The focus is to directly intervene in the conversation with textual responses that counter the hate content and prevent it from further spreading. The aim of this research is to analyse the quality of counter-narrations in combating against online hate speech using automated generation-techniques like pre-trained Language Models. We conduct experiments on a diverse corpus of online speech and the responses they generate and evaluate the effect and quality of a novel approach on the dialogue.  

\end{abstract}
