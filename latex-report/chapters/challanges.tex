\chapter{Implementation Challanges}
There are several challenges for implementing this protocol.
As we are going to use UDP as a transport wrapper, it
is hard to maintain a steady connection. A major problem
arises when a client stays behind the NAT. In the case of
TCP connection, NAT keeps port binding until connection
terminates. However UDP provides a connectionless wrapper,
and there is no concept of connection termination. So, there is
no defined rule about how long the NAT will keep UDP port
binding. In general, they hold binding at least for 5 seconds.
So server needs to send data within 5 seconds. We can resolve
this issue by sending polling packet to the server in regular
interval.

\paragraph{} The next challenge is the network stack related issues. By default, existing implementation doesn’t provide any method
to directly send a packet to a network interface (We have
observed this in Linux kernel). We can solve this issue by
change kernel a bit or change the routing table a bit (like
what is done in MPTCP). Presently, the protocol is under
development and we are testing several systems related issues
for running it over low-cost IoT devices.
%In our proposed scheme, we have devised mechanism to improve the good-put of E-TCP. In our effort to achieve this goal we have to lose out on the bandwidth utilization of the E-TCP. Based on our experiments we observe that, TCP-Vegas gives better good put than TCP-Reno. From this observation, we decide to add RTT based congestion control algorithm to E-TCP. We have two options to carry out this job, 1) Use an existing RTT based congestion control algorithm like TCP-Vegas, 2) Devise our own congestion detection algorithm. We propose two different schemes namely \textit{EV-TCP} and \textit{EV2-TCP}. We combine E-TCP and TCP Vegas to devise EV-TCP. EV2-TCP is based on queuing delay detection scheme. We will discuss EV-TCP and EV2-TCP in this chapter.

%\section{EV-TCP}
%We combine TCP Vegas and E-TCP for better good put. There are number of approaches for RTT based congestion control like Wang and Crowcroft's DUAL algorithm , Jain's CARD (Congestion Avoidance using Round-trip Delay)  or Wang and Crowcroft's Tri-S scheme . But among them, we choose TCP Vegas because it has been implemented successfully in several network stacks. TCP Vegas is a modification of TCP Reno, to detect congestion before any packet loss occurs.

% \paragraph{} We didn't change the actual E-TCP's congestion control algorithm. We just patched a modified RTT based congestion control mechanism from TCP Vegas.% This mechanism has been modified to implement in E-TCP. 

%\paragraph{} We do not change the actual E-TCP's congestion control algorithm. We have modified the congestion signalling mechanism by adding RTT based congestion detection technique. 



%\paragraph{Congestion Signalling: } Like E-TCP, basic congestion control mechanism of EV-TCP is based on selective acknowledgement. EV-TCP includes three extra parameters namely \textit{cc\_seq}, \textit{h\_seq}, \textit{bitmap}. The functionality of these parameters are similar to E-TCP, so we are not elaborating them here. In next part we will discuss signalling procedure at each end.

