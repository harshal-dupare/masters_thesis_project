\chapter{Introduction}

\hspace{1cm}Optimization problems are of great interest academically along with being core to the functioning of today's society by having application in variety of fields such as social studies, economics, finance, agriculture, automotive, engineering, computer science, networking and many others.

In general there can be more than one objective and constraint in the formulation of optimization problems. Optimization problems are classified into 2 broad categories namely \textit{Single Objective Optimization Problems}(SOOP) and \textit{Multi-Objective Optimization Problems} (MOOP), the latter can be considered a super-set of the former. Our focus will be on a special subset of MOOP where the \textit{Unknown}/\textit{Decision Variable} is a \textit{function} lets call them fMOOP. fMOOP are of great interest because of their vast applications in modeling the relation between 2 sets, for e.g. all of the Neural Network so heavily researched today fall into this category with there myriads of applications. 

Generally when we have a function, we want it to have some good properties like being correct, easy to compute, easy to store, behave nicely over the input set, or any other desirable properties based on the context. One such very important property for most of such functions of interest is being \textit{Well-Conditioned}, meaning small changes in the input should result in too big changes in the output. We focus on this problem namely the problem to find function solutions of fMOOP which are well-conditioned.

Goal of this study is to analyze the implications of restricting the condition number for solutions of fMOOP, and study how the existing methodologies perform on such problems and device new methodologies for finding well-conditioned solutions to fMOOP.

% * How important are optimization methods and their vaste applications\\
% * Highlight importance of subsections of optimization methods where we are looking for functions belonging to certain family, and give eg\\
% * Then highlight how important it is for those functions to be stable for small changes in input, i.e. locality of these problems, and show if thats not maintained then how it can lead to bad functions and how they can be exploited \\
% * Hence its impostant to model this problem wrt MOOP and study it\\
\section{Preliminaries}
\subsection{Definitions}

\textbf{\textit{Aggregation Scheme:}} Given a value function $value(\cdot)$ over $x$, for $x \in X$, a method to aggregate those values to one value. Denoted as follows:
\begin{equation}
\underset{\forall x\in X}{Agg}\hspace{1mm} value(x)
\end{equation}
e.g. Expectation over given probability distribution $p(\cdot)$ over $X$.
\begin{equation}
\underset{\forall x\in X}{Agg}\hspace{1mm} value(x) = \mathlarger{\mathlarger{\int}}_{\forall x\in X} value(x)p(x)\hspace{1mm}dx
\end{equation}
e.g. Exponential decaying average, given a index-function $t:X\to[|X|]$, where $[n] = \{0,1,2,...,n-1\}$.
\begin{equation}
\underset{\forall x\in X}{Agg}\hspace{1mm} value(x) = \mathlarger{\mathlarger{\sum}}_{\forall x\in X} \lambda^{t(x)}value(x)
\end{equation}
\newline
\textbf{\textit{Absolute Condition Number:}} \textit{Absolute Condition Number} is defined for a function $f: X \to Y$, where $X$ and $Y$ have a norm $||.||$ is defined over them. Denoted as follows:

\begin{equation} \label{def_cond_abs}
cond_{abs}(f) = \underset{\epsilon \to 0}{lim}\underset{||\delta x||\le \epsilon}{sup} \frac{||\delta f(x)||}{||\delta x||}
\end{equation}
\newline
\textbf{\textit{Relative Condition Number:}} \textit{Relative Condition Number} is defined for a function $f: X \to Y$, where $X$ and $Y$ have a norm $||.||$ is defined over them. Denoted as follows:

\begin{equation} \label{def_cond_rel}
cond_{x}(f) = \underset{\epsilon \to 0}{lim}\underset{||\delta x||\le \epsilon}{sup} \frac{||\delta f(x)||/||f(x)||}{||\delta x||/||x||}
\end{equation}
\newline
\textbf{\textit{Matrix Transform Norm:}} Given a Matrix Transform $T^{m\times k}: \mathbb{R}^{n\times m} \to \mathbb{R}^{n\times k}$ and a norm $||\cdot||$ defined over $\mathbb{R}^{n\times m}$ and $\mathbb{R}^{n\times k}$ we define norm of the matrix transform as 

\begin{equation}
   ||T^{m\times k}|| = \underset{X \in \mathbb{R}^{n\times m}/\{\Bar{0}\}}{sup} \frac{||X\times T^{m\times k}||}{||X||}
\end{equation}
\textit{Note}: Whenever the dimensions of the matrix in the context are already defined or are obvious from the context we will omit the superscript denoting its dimension.
\subsection{Problem} \label{prob_def}

\textbf{Given:}\newline
Domain $X_1\cup X_2 \subseteq X$ and their associated Ranges $Y_1\cup Y_2 \subseteq Y$, where $X_i\equiv Y_i$ such that for each $x \in X_i$ we have a corresponding $y \in Y_i$ i.e. $y\equiv x$.\newline
Aggregation schemes $\underset{\forall x\in X_1}{Agg_1}$ defined over $\mathcal{L}: Y \times Y \to \mathbb{R}$ call $\mathcal{L}(\cdot,\cdot)$ the \textit{Loss} function and $\underset{\forall x\in X_2}{Agg_2}$ defined over $\mathcal{R}: Y \times Y \to \mathbb{R}$ call $\mathcal{R}(\cdot,\cdot)$  the \textit{Reward} function.\newline
A scalar $\alpha \in \mathbb{R^+}$ and subset $X_3 \subseteq X$.\newline
For a given $\mathcal{F}$ a family of functions $f: X \to Y$ we need to find a function $f \in \mathcal{F}$ which solves the following \newline\newline \textbf{Functional Multi-Objective Optimization Problem:}

\begin{equation} \label{opt_loss}
\underset{f\in \mathcal{F}}{min}\hspace{1mm} \underset{\forall x\in X_1, y \in Y_1, y \equiv x}{Agg_1} \hspace{1mm} \mathcal{L}(y,f(x))
\end{equation}
\begin{equation} \label{opt_reward}
\underset{f\in \mathcal{F}}{max}\hspace{1mm} \underset{\forall x\in X_2, y \in Y_2, y \equiv x}{Agg_2} \hspace{1mm} \mathcal{R}(y,f(x))
\end{equation}
subject to one of the following
\begin{equation} \label{cond_abs}
cond_{abs}(f) \le \alpha 
\end{equation}
\begin{equation} \label{cond_rel}
\underset{x \in X_3}{max}\hspace{1mm} cond_{x}(f) \le \alpha 
\end{equation}
\begin{equation} \label{cond_abs_min}
\underset{f\in \mathcal{F}}{min}\hspace{1mm}cond_{abs}(f)
\end{equation}
\begin{equation} \label{cond_rel_min}
\underset{f\in \mathcal{F}}{min}\hspace{1mm} \underset{x \in X_3}{max}\hspace{1mm}cond_{x}(f)
\end{equation}
\newline It is very important to mention that the above problem formulation highly depends on the nature of the mapping $f$ and the family $\mathcal{F}$ that it belongs to. For example if $f$ models a function for which we expect smooth change in its value w.r.t. its domain then its a reasonable formulation, and if there are no reasons to believe that it should be smooth over its domain then its not a reasonable formulation. Similarly for a chosen family $\mathcal{F}$ to model $f$, if $\mathcal{F}$ doesn't have smooth functions belonging to it then its not a reasonable formulations irrespective of the nature of the function that $f$ is trying to model.
\newline\newline
\textbf{\textit{Note}}: Here we restricted the problem to 2 objectives and only 1 condition which majorly characterizes this problem, but in general we can have arbitrary number of objectives and other conditions along with at least 1 of the \ref{cond_abs}, \ref{cond_rel} conditions.\newline
What we mean by minimizing over a functions $f \in \mathcal{F}$ of a family of functions $\mathcal{F}$ is that we minimize over the parameters of that family of function $\mathcal{F}$.

% \newpage 
\section{Motivation and Applications}

\subsection{Linear Factor Model}
The problems defined in the former section was motivated from the problem on linear factor models with added constraint for the returns-to-factor matrix be orthonormal columns. We describe the problem below.\newline
\textit{Given}:\newline Returns data for $n$ time series $R^{n\times 1}_t \in \mathbb{R}^{n\times 1}$, for $t\in [T]$ \newline
and denote $R^{n\times 1}_{t\times d} = [R^{n\times 1}_{t-1},R^{n\times 1}_{t-2},...,R^{n\times 1}_{t-d}] \in \mathbb{R}^{n\times d}$\newline
We need to design $F^{n\times k}_t \in \mathbb{R}^{n\times k}$ as a function of $R^{n\times d}_{t\times d}$, i.e. $F^{n\times k}_t = f(R^{n\times d}_{t\times d})$ and from that we can derive $P^{n \times 1}_{t} = g(F^{n\times k}_t)$ such that $P^{n \times 1}_{t} \approx R^{n \times 1}_{t}$ which we can measure by \textit{Average} aggregation scheme over the loss function $\mathcal{L}(y,\hat{y}) = ||y-\hat{y}||_2$ which we need to minimize and another \textit{Average} aggregation scheme over the return function $\mathcal{R}(y,\hat{y}) = \frac{\langle y,\hat{y}\rangle}{||y||_2||\hat{y}||_2}$ defined over the testing period of $t \in [T,T+S]= \{T,T+1,...,T+S-1\}$ which we need to maximize.

\begin{equation} \label{lfm_diagram}
\begin{tikzcd}
R^{n\times d}_{t \times d} \arrow[rr, "f" description] \arrow[r, no head]& {}& F^{n\times k}_{t} \arrow[d, "g" description] \\
R^{n\times 1}_t \arrow[rr, "\mathcal{L}" description, Rightarrow, no head] \arrow[rd, dashed] && P^{n \times 1}_{t} \arrow[ld, dashed]\\
& \mathcal{R} &
\end{tikzcd}
\end{equation}
\newline
What makes this model linear is that the function $f(\cdot)$ and $g(\cdot)$ are linear w.r.t. their argument. In this case it can be written as
\begin{equation} \label{f_for_lfm}
F^{n\times k}_t = f(R^{n\times d}_{t\times d}) = R^{n\times d}_{t\times d}\times A^{d\times k}
\end{equation}

\begin{equation} \label{g_for_lfm}
P^{n \times 1}_{t} = g(F^{n\times k}_t) = F^{n\times k}_t \times \beta^{k \times 1}
\end{equation}
\newline
And the orthonormal condition requires
\begin{equation} \label{orthonormal_eq}
 A^{d\times k}(A^{d\times k})^{\top} = I^{d\times d}
\end{equation}
\newline
From the \ref{orthonormal_eq} we can infer that $(A^{d\times k})^T$ belongs to the set of right inverses of $A^{d\times k}$. \newline Further if we relax the \ref{orthonormal_eq} to bounds on condition number by $\alpha \ge 1$
\begin{equation}
    \underset{R^{n\times d}_{t\times d} \in \mathbb{R}^{n\times d}}{max}\hspace{1mm} cond_{R^{n\times d}_{t\times d}}(f) = \underset{\epsilon \to 0}{lim}\underset{||\delta R^{n\times d}_{t\times d}||\le \epsilon}{sup} \frac{||\delta R^{n\times d}_{t\times d}\times A^{d\times k}||}{||\delta R^{n\times d}_{t\times d}||}  \frac{||F^{n\times k}_t \times (A^{d\times k})^{\top} ||}{|| F^{n\times k}_t||}\le \alpha
\end{equation}
which implies\newline
\begin{equation}
    \underset{R^{n\times d}_{t\times d} \in \mathbb{R}^{n\times d}}{max}\hspace{1mm} cond_{R^{n\times d}_{t\times d}}(f) \le ||A^{d\times k}|| ||(A^{d\times k})^{\top} ||\le \alpha
\end{equation}
\newline
Note that $\alpha=1$ contains the set which satisfies \ref{orthonormal_eq} equation since 
\begin{equation}
1= || I^{d\times d}||= || A^{d\times k}(A^{d\times k})^{\top} ||  \le || A^{d\times k}||||(A^{d\times k})^{\top} || \le \alpha
\end{equation}\newline
So any $\alpha \ge 1$ will contain the set specififed by the condition \ref{orthonormal_eq}\newline
Combining all of that together gives us the \textbf{fMOOP} formulation of this problem as

\begin{equation} \label{lfm_loss}
\underset{\beta^{k \times 1}\in \mathbb{R}^{k\times 1}, A^{d\times k} \in \mathbb{R}^{d\times k}}{min}\hspace{1mm} \frac{1}{T-d+1} \mathlarger{\mathlarger{\sum}}_{\forall t\in [d,T]} \hspace{1mm} ||R^{n\times 1}_t-R^{n\times d}_{t\times d}\times A^{d\times k} \times \beta^{k \times 1}||_2
\end{equation}
\begin{equation} \label{lfm_reward}
\underset{\beta^{k \times 1}\in \mathbb{R}^{k\times 1}, A^{d\times k} \in \mathbb{R}^{d\times k}}{max}\hspace{1mm} \frac{1}{S} \mathlarger{\mathlarger{\sum}}_{\forall t\in [T,T+S]} \hspace{1mm} \frac{\langle R^{n\times 1}_t,R^{n\times d}_{t\times d}\times A^{d\times k} \times \beta^{k \times 1}\rangle}{||R^{n\times 1}_t||_2||R^{n\times d}_{t\times d}\times A^{d\times k} \times \beta^{k \times 1}||_2}
\end{equation}
subject to
\begin{equation} \label{lfm_cond}
\underset{R^{n\times d}_{t\times d} \in \mathbb{R}^{n\times d}}{max}\hspace{1mm} cond_{R^{n\times d}_{t\times d}}(f) \le \alpha
\end{equation}
\newline
\subsection{Principle Time Series}
Another task related to time series is to represent a set of $n$ time series by less number of time series $k<n$. Which is some way is an application of the Linear Factor Models.\newline
\textit{Given}:\newline The definition of $R^{n\times 1}_t$ and  $R^{n\times 1}_{t\times d}$ are same as before, but here we need to design latent time series denote them by $F^{k\times 1}_t \in \mathbb{R}^{k\times 1}$ as a function of $R^{n\times d}_{t+1\times d}$, i.e. $F^{k\times 1}_t = f(R^{n\times d}_{t+1\times d})$ which reduces the $n$ time series observed from time $t-d$ to time $t$ to set of $k$ time series at time $t$ such that its possible to sufficiently recover the $n$ time series at time $t$ by another function $ g(F^{k\times 1}_t) = P^{n \times 1}_{t}$ such that $P^{n \times 1}_{t} \approx R^{n \times 1}_{t}$ which we can measure by \textit{Average} aggregation scheme over the loss function $\mathcal{L}(y,\hat{y}) = ||y-\hat{y}||_2$, which we need to minimize.

\begin{equation} \label{pts_diagram}
\begin{tikzcd}
R^{n\times d}_{t+1 \times d} \arrow[rr, "f" description] \arrow[r, no head]& {}& F^{k\times 1}_{t} \arrow[d, "g" description] \\
R^{n\times 1}_t \arrow[rr, "\mathcal{L}" description, Rightarrow, no head] \arrow[rd, dashed] && P^{n \times 1}_{t} \arrow[ld, dashed]\\
& \mathcal{R} &
\end{tikzcd}
\end{equation}\newline 
Further we can restrict the functions $f(\cdot)$ and $g(\cdot)$ to be linear w.r.t. their argument. In that case it can be written as

\begin{equation} \label{f_for_pts}
F^{k\times 1}_t = f(R^{n\times d}_{t+1\times d}) = A^{k\times n}\times R^{n\times d}_{t+1\times d}\times B^{d\times 1}
\end{equation}

\begin{equation} \label{g_for_pts}
P^{n \times 1}_{t} = g(F^{k\times 1}_t) = C^{n \times k} \times F^{k\times 1}_t
\end{equation}
\newline
For small changes in our time series data $\delta R^{n\times d}_{t+1\times d}$ we expect that there is small changes in the latent time series $\delta F^{k\times 1}_t$, which we can model by bounding the absolute condition number by a scalar $\alpha$.

\begin{equation}
    cond_{abs}(f) = \underset{\epsilon \to 0}{lim}\underset{||\delta R^{n\times d}_{t+1\times d}||\le \epsilon}{sup} \frac{||A^{k\times n}\times \delta R^{n\times d}_{t+1\times d}\times B^{d\times 1}||}{||\delta R^{n\times d}_{t+1\times d}||}
\end{equation}\newline
Combining all of that together gives us the \textbf{fMOOP} formulation of this problem as
\begin{equation} \label{pts_loss}
\underset{A\in \mathbb{R}^{k\times n}, B \in \mathbb{R}^{d\times 1}, C \in \mathbb{R}^{n\times k}}{min}\hspace{1mm} \frac{1}{T-d+1} \mathlarger{\mathlarger{\sum}}_{\forall t\in [d,T]} \hspace{1mm} ||R^{n\times 1}_t-C^{n \times k} \times A^{k\times n}\times R^{n\times d}_{t+1\times d}\times B^{d\times 1}||_2
\end{equation}\newline
subject to
\begin{equation} \label{lfm_cond}
cond_{abs}(f) \le \alpha
\end{equation}
\newline

\newpage
\subsection{Defense Against Adversarial Attacks on Neural Networks}

Say we are given a trained neural network $\mathcal{N}: X \to Y$ which has learnt a mapping from $X$, the input set to $Y$, the output set. If the underlining mapping that $\mathcal{N}$ was modeled to learn was smooth w.r.t. its input then we expect that for a well trained network $\mathcal{N}$ for any input $x\in X$ and for small enough $\delta x \in X$  $\delta x$ the output of the network wont change much, lets say that it won't change more than a constant $\alpha > 0$ times the norm of $x$ i.e. $||\mathcal{N}(x+\delta x) - \mathcal{N}(x)|| = ||\delta \mathcal{N}(x)||\le \alpha ||\delta x||$. Which we can reformulate as follows

\begin{equation}
    cond_{abs}(\mathcal{N}) = \underset{\epsilon \to 0}{lim}\underset{||\delta x||\le \epsilon}{sup} \frac{||\delta \mathcal{N}(x)||}{||\delta x||} \le \alpha
\end{equation}\newline 
Most of the networks which are being trained today don't account for such constraints giving the way to Adversarial attacks on them, which exploit this drawbacks of the network to force them to output unreasonably wrong value.
\begin{equation}
\begin{tikzcd}
{(x,y,y')} \arrow[rr, "\mathcal{A}_{\epsilon}" description] \arrow[rrd, no head, dashed] \arrow[d, no head, dashed] &  & \delta x \arrow[d, no head, dashed]\\
x \arrow[d, "\mathcal{N}" description] &  & x+\delta x \arrow[d, "\mathcal{N}" description] \\
y& & y'
\end{tikzcd}
\end{equation}\newline
Here $\mathcal{A}_{\epsilon}$ the adversary which takes the input $(x,y,y')$ being the input data $x$, the associate label $y$, and the expected forced wrong output $y'$ and outputs the required perturbation $\delta x$, such that $||\delta x|| \le \epsilon$. When $\delta x$ is added to original input data the new malicious input $x_m = x + \delta x$ forces the network $\mathcal{N}$ to output $y'$ instead of $y$ whereas it would have given output of $y$ on the input $x$. Note that the adversary prefers smaller values of $\epsilon$, since that implies that it can generate malicious input with as little changes as possible.\newline
This problem can be avoided if we can train the network to also minimize the absolute condition number, $cond_{abs}(\mathcal{N})$ or minimize the maximum relative condition number over a subset $X_{I} \subseteq X$, $\underset{x \in X_{I}}{max}\hspace{1mm} cond_{x}(\mathcal{N})$. We can also provide the constrains on their maximum values and model accordingly.\newline
So if the network has been trained such that $||\delta \mathcal{N}(x)||\le \alpha ||\delta x||$ holds, then for the adversary $\mathcal{A}_{\epsilon}$ to make the network's $\mathcal{N}$ output perturb by $||\delta \mathcal{N}(x)||$ it will have to change the input $x$ value perturbation of norm more than $\frac{||\delta \mathcal{N}(x)||}{\alpha} \le ||\delta x||$, which will not be possible for an adversary with $\epsilon < \frac{||\delta \mathcal{N}(x)||}{\alpha}$. Hence training with such constraints will force the adversary to make large changes to the input for designing an malicious input, which is unfavourable for the adversary.