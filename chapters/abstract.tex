% \vspace{2in}

\begin{abstract}
 This thesis defines the problem of Functional Multi-Objective Optimization Problem (fMOOP) in which we are looking for functions as solutions of Multi-Objective Optimization Problem, where one of the objective/constraint is to minimize/bound the condition number of such a solution function. We give motivation for studying such problems and why they are important. Then we survey the different methodologies for solving MOOP in general and summarize a few results on the condition number of matrices and function. Then we address some of the important questions regarding such problems and analyze different methods for finding well-condition solutions to fMOOP.  Based on the analysis we will select a few methods and compare their experimental results.
% \newline\newline Will be completed when other contents are added

% \begin{itemize}
%     \item what is my research problem and objectives/goal [Done]
%     \item what methods did/can I used [Done]
%     \item what were the key results and arguments
%     \item conclusion
% \end{itemize}



% \paragraph{} Growing Hate-content on online platforms is a serious problem and has given rise to numerous  discussions about the need for regulation and monitoring of posted content. However using stringent measures such as account suspension or removal of the post somehow is a threat to the doctrine of free speech.   

% \paragraph{} Researchers worldwide have found that counter -narratives can effectively neutralize hate speech and incite constructive dialogue. The focus is to directly intervene in the conversation with textual responses that counter the hate content and prevent it from further spreading. The aim of this research is to analyse the quality of counter-narrations in combating against online hate speech using automated generation-techniques like pre-trained Language Models. We conduct experiments on a diverse corpus of online speech and the responses they generate and evaluate the effect and quality of a novel approach on the dialogue.  

\end{abstract}
